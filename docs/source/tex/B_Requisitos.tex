\apendice{Especificación de Requisitos}

\section{Introducción}

En este apéndice se presenta la especificación de requisitos del proyecto, que incluyen los objetivos generales, el catálogo de requisitos y la especificación de requisitos. La especificación de requisitos es una parte fundamental del desarrollo de software, ya que permite definir de forma clara y precisa las necesidades y expectativas del cliente.




\section{Objetivos generales}

\section{Catálogo de requisitos}
Requisitos funcionales y no funcionales del proyecto, que se han clasificado en los siguientes grupos:

\tablaSmall{Tabla de requisitos funcionales}{l p{6em} p{15em} c}{requisitosfuncionales}{Número & Requerimiento & Descripción & Prioridad \\}{ 
RF1 & Facilidad de uso & Las interfaces de usuario deben de ser simples y adecuadas para todo tipo de usuarios. & Alta \\
} 

\tablaSmall{Tabla de requisitos no funcionales}{l p{6em} p{15em} c}{requisitosnofuncionales}{Número & Requerimiento & Descripción & Prioridad \\}{ 
RNF1 & Facilidad de uso & Las interfaces de usuario deben de ser simples y adecuadas para todo tipo de usuarios. & Alta \\
} 

\section{Especificación de requisitos}

Casos de uso
Hacer 5?

% Caso de Uso 1 -> Consultar Experimentos.
\begin{table}[p]
	\centering
	\begin{tabularx}{\linewidth}{ p{0.21\columnwidth} p{0.71\columnwidth} }
		\toprule
		\textbf{CU-1}    & \textbf{Ejemplo de caso de uso}\\
		\toprule
		\textbf{Versión}              & 1.0    \\
		\textbf{Autor}                & César Rodríguez Villagrá \\
		\textbf{Requisitos asociados} & RF-xx, RF-xx \\
		\textbf{Descripción}          & La descripción del CU \\
		\textbf{Precondición}         & Precondiciones (podría haber más de una) \\
		\textbf{Acciones}             &
		\begin{enumerate}
			\def\labelenumi{\arabic{enumi}.}
			\tightlist
			\item Pasos del CU
			\item Pasos del CU (añadir tantos como sean necesarios)
		\end{enumerate}\\
		\textbf{Postcondición}        & Postcondiciones (podría haber más de una) \\
		\textbf{Excepciones}          & Excepciones \\
		\textbf{Importancia}          & Alta o Media o Baja... \\
		\bottomrule
	\end{tabularx}
	\caption{CU-1 Nombre del caso de uso.}
\end{table}