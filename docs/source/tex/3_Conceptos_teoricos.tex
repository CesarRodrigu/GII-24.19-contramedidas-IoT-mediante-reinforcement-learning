\capitulo{3}{Conceptos teóricos}
En esta capítulo se van explicar los conceptos teóricos relacionados con el proyecto, que son necesarios para la comprensión del mismo.

\subsection{Aprendizaje por refuerzo}
El aprendizaje por refuerzo o \textit{Reinforcement Learning} (RL) es una rama del aprendizaje automático que se centra en cómo los agentes deben de tomar decisiones en un entorno, para maximizar la recompensa asociada a sus acciones y estado del entorno. A diferencia del aprendizaje supervisado, donde se dispone de un conjunto de datos etiquetados, en el aprendizaje por refuerzo el agente aprende a través de la interacción con el entorno, recibiendo recompensas o penalizaciones en función de las acciones que realiza. Su aprendizaje se basa en la prueba y error, donde el agente aprende a través de la experiencia acumulada en el entorno. 
Los principales elementos del aprendizaje por refuerzo son:
\begin{itemize}
	\item \textbf{Agente:} Es el ente que se encarga de la toma de decisiones y aprende a través de la interacción con el entorno a lo largo del tiempo.
	\item \textbf{Entorno:} Es el contexto en el que el agente se encuentra, incluyendo todos los elementos con los que se relaciona.
	\item \textbf{Estado:} Es una instantánea en un momento dado, que puede incluir información sobre el estado actual del agente y del entorno.
	\item \textbf{Acción:} Es una decisión tomada por el agente que afecta al estado del entorno.
	\item \textbf{Política:} Es una estrategia que define cómo el agente selecciona acciones en función del estado actual del entorno.
	\item \textbf{Recompensa:} Es una señal que indica al agente la calidad de la acción tomada en un estado determinado. El objetivo del agente es maximizar la recompensa acumulada a lo largo del tiempo.
\end{itemize}

El aprendizaje por refuerzo se utiliza en una amplia variedad de aplicaciones, como juegos, robótica, optimización de procesos, entre otros.

Se usa en situaciones donde:
\begin{itemize}
	\item Las acciones afectan el estado futuro del entorno
	\item No se dispone de un conjunto de datos etiquetados
	\item Se tiene retroalimentación escasa o tardía de la acción que se ha tomado
	\item El agente debe aprender a través de la experiencia acumulada.
\end{itemize}
Es especialmente útil en entornos muy variables o dinámicos, donde se tienen que tomar acciones secuencialmente y se necesita una adaptación a los cambios en el entorno.


\subsection{Redes de computadoras}
Una red de computadoras es un conjunto de dispositivos interconectados que pueden comunicarse entre sí. Permiten el intercambio de mensajes y recursos entre los elementos de la red. Actualmente las redes de computadoras están muy extendidas y son fundamentales para el funcionamiento de Internet y de muchas aplicaciones y servicios en línea.
Las redes se pueden clasificar en diferentes tipos según su alcance, los más comunes son:
\begin{itemize}
	\item \textbf{Redes de área local (LAN):} Conectan dispositivos en un área geográfica limitada, como una oficina o un edificio.
	\item \textbf{Redes de área amplia (WAN):} Conectan dispositivos en áreas geográficas más amplias, como ciudades o países.
\end{itemize}

Los elementos principales de una red de computadoras son:
\begin{itemize}
	\item \textbf{Dispositivos finales:} Son los dispositivos que se conectan a la red, los cuales se quieren comunicar con otros dispositivos finales. Los dispositivos finales pueden ser computadoras, teléfonos móviles, impresoras, cámaras, sensores, etc.
	\item \textbf{Dispositivos de red:} Son los dispositivos que permiten la comunicación entre los dispositivos finales, como routers, switches, etc.
	\item \textbf{Medios de transmisión:} Son los canales a través de los cuales se transmiten la información entre dispositivos, como cables Ethernet, fibra óptica, Wi-Fi, etc.
\end{itemize}

La comunicación entre dispositivos se realiza a través de paquetes de datos que viajan por los medios de transmisión. Los paquetes incluyen información sobre el origen, destino, protocolo, y contenido del mensaje entre otros. Es importante tener en cuenta el tamaño de los paquetes, siendo necesario en algunas ocasiones dividir los mensajes en varios paquetes para su transmisión.


\subsection{Internet de las cosas}

\subsection{Ataques informáticos}

\subsubsection{Tipos de ataques}

\subsubsection{Contramedidas}

\subsection{Ciberseguridad}





