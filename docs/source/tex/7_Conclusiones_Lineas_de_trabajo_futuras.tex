\capitulo{7}{Conclusiones y Líneas de trabajo futuras}

\section{Conclusiones}
El proyecto ha conseguido desarrollar un agente, que mediante el uso de aprendizaje por refuerzo es capaz de mitigar ataques de denegación de servicio en un entorno de simulación basado en dispositivos IoT.

También en la aplicación web desarrollada es importante la visualización de los datos y de la documentación del proyecto, permitiendo a los usuarios una mayor compresión del proyecto.

\section{Líneas de trabajo futuras}
Para la continuación del proyecto se han planteado varias líneas de trabajo futuras, que buscan mejorar el rendimiento del agente y la funcionalidad de la aplicación web. 

\subsection{Mejoras en el agente de aprendizaje por refuerzo}
Las principales líneas de trabajo son:
\begin{itemize}
    \item \textbf{Mejora en la recompensa:} Optimizar la función de recompensa, añadiendo más variables para que el agente tenga más información relevante y así poder aprender.
    \item \textbf{Mitigación de más ataques:} Añadir al agente la posibilidad de mitigación de más tipos de ataques como:
    \begin{itemize}
        \item Escaneo de puertos (PortScan)
        \item Escaneo de sistema operativo (OSScan)
    \end{itemize}
    \item \textbf{Desarrollar e implementar un clasificador de ataques previo al agente:} Añadiendo un clasificador de ataques que permita clasificar los ataques recibidos en diferentes categorías, para poder aplicar diferentes estrategias de mitigación según el tipo de ataque. En caso que el clasificador no pueda clasificar el ataque con suficiente confianza, se podría aplicar una estrategia de mitigación genérica por el agente.
\end{itemize}

\subsection{Mejoras en la aplicación web}
Las principales líneas de trabajo son:
\begin{itemize}
    \item \textbf{Mejora en el entorno de simulación:} Llevar la simulación a un entorno de simulación de tráfico, como GNS3, en el que se implemente el agente y se pueda ver y evaluar cual es su desempeño en un entorno más complejo.
    \item \textbf{Optimización del rendimiento:} Mejorar la eficiencia de la aplicación web, reduciendo los tiempos de carga y optimizando el uso de recursos.
    \item \textbf{Mejora de la interfaz de usuario:} Realizar cambios en el diseño y la usabilidad de la aplicación para facilitar la interacción del usuario. En cuanto a la parte técnica, se plantea la sustitución de los estilos de Bootstrap 5 por un framework de frontend moderno como \textit{React}, \textit{Vue.js} o \textit{Angular}, lo que permitiría una interfaz más dinámica, modular y fácil de mantener.
    \item \textbf{Ampliación de funcionalidades:} Incluir nuevas características en la aplicación web, como la posibilidad de ajustar los parámetros de entrenamiento del agente desde la aplicación web, permitiendo a los usuarios personalizar el comportamiento del agente según sus necesidades.
    \item \textbf{Mejora de la seguridad:} Implementar medidas adicionales para proteger la aplicación web y los datos de los usuarios, como la autenticación de dos factores y el cifrado de las comunicaciones mediante el protocolo HTTPS entre Cloudflare y el servidor de AWS.
    \item \textbf{Ampliar las pruebas:} Incrementar las pruebas de software para asegurar la calidad y estabilidad de la aplicación, haciendo énfasis en las pruebas de Selenium, para asegurar el correcto funcionamiento de la aplicación web en diferentes navegadores y dispositivos.
    \item \textbf{Mejora de la documentación del proyecto:} Ampliar y mejorar la documentación del proyecto, incluyendo guías más detalladas de usuario, manuales de instalación y uso en diferentes idiomas, mejorando las ya existentes en español y creando nuevas en inglés.
\end{itemize}
