\apendice{Plan de Proyecto Software}

\section{Introducción}
En este apéndice se pretende mostrar el plan de proyecto software que se ha seguido durante el desarrollo del proyecto.

La implementación del plan de proyecto siguiendo una metodología ágil, ha permitido una mayor flexibilidad y adaptabilidad a los cambios que han surgido durante el desarrollo. En este caso, se ha utilizado principalmente la metodología de Kanban, debido a la flexibilidad que presenta en la organización y gestión del trabajo.

En cuanto a la estructura de este anexo, se explica la planificación temporal del proyecto y el estudio de viabilidad, tanto económica como legal. 

\subsection{Kanban}
La metodología Kanban~\cite{JuliaMartins2025} se ha utilizado para gestionar el flujo de trabajo del proyecto. En el que se usa el tablero de Kanban perdeterminado en los GitHub Projects, donde se han usado las siguientes columnas:
\begin{itemize}
    \item \textbf{Backlog:} En esta columna se encuentran todas las tareas que se han propuesto para el proyecto, pero que aún no se han comenzado a desarrollar, limitado a 5 tareas.
    \item \textbf{Ready:} En esta columna se encuentran todas las tareas que estaban en el \textit{Backlog}, y que están listas para ser desarrolladas.
    \item \textbf{In Progress:} En esta columna se encuentran todas las tareas que están en proceso de desarrollo, limitado a 3 tareas.
    \item \textbf{In Review:} En esta columna se encuentran todas las tareas que han sido completadas y están en revisión por parte de los autores del proyecto, limitado a 5 tareas.
    \item \textbf{Done:} En esta columna se encuentran todas las tareas que ya han sido completadas.
\end{itemize}

Todas las tareas que se han terminado quedan registradas en la columna \textit{Done}, y se han ido moviendo a lo largo de las columnas según se ha ido avanzando en el desarrollo del proyecto. En cada tarea, si forma parte de desarrollo de una parte de código o documentación, quedan registrado en la tarea una rama de desarrollo de GitHub, y una pull request asociada a la tarea en la que se han incorporado los cambios realizados a la rama principal. En cada pull request se han incluido comentarios creados pro GitHub Copilot, en la que se ha incluido una descripción de los cambios realizados, y una lista de los archivos que han sido modificados, y una revisión de errores o propuestas de mejora para implementar antes de la incorporación de cambios.

\section{Planificación temporal}
Cada semana, si es necesario se actualiza el tablero del proyecto.
Se han planificado cada semana una reunión de seguimiento del proyecto con los tutores, en la que se han revisado las tareas que se han realizado a lo largo de la semana. En la que se plantean propuestas de mejora y correcciones a las tareas realizadas.

Las reuniones semanales empezaron el día 6 de febrero de 2025 hasta el día 5 de junio de 2025, en el que cada jueves a las 10:00 se han realizado las reuniones, a excepción de festivos y vacaciones.

\subsection{Burn up}
El gráfico de Burn up muestra el progreso del proyecto a lo largo del tiempo, en el que se puede ver la cantidad de trabajo completado y el trabajo restante. En el eje horizontal se muestra el tiempo, y en el eje vertical se muestra la cantidad de trabajo completado. La línea morada representa el trabajo completado, y la línea verde representa el trabajo restante. El área entre las dos líneas representa el trabajo pendiente.
% #TODO Actualizar imágenes
\imagen{burn_up}{Gráfico de Burn up del proyecto}

\subsection{Diagrama de estado}
El diagrama de estado del proyecto muestra los diferentes estados en los que se encuentran actualmente el estado de las tareas del proyecto. Como el desarrollo del proyecto ya se ha terminado, el estado de las tareas se encuentra en "Done".
\imagen{project_status_chart}{Diagrama de estado del proyecto}

\subsection{Diagrama de trabajo por tema}
El diagrama de trabajo por tema muestra los diferentes temas en los que se ha trabajado a lo largo del proyecto, y la cantidad de trabajo realizado en cada uno de ellos. En el eje horizontal se muestran los diferentes temas, y en el eje vertical se muestra la cantidad de trabajo realizado.
Donde de izquierda a derecha se pueden ver: Lecturas de artículos,  Procesos de CI, Desarrollo de la web, con la api y el frontend, Documentación, Desarrollo y entrenamiento del agente.



\imagen{project_stimated}{Diagrama de trabajo por tema del proyecto}

El orden de los temas que han requerido más trabajo ha sido el siguiente:
\begin{enumerate}
    \item \textbf{Desarrollo y entrenamiento del agente \#7:} El tema que más trabajo ha requerido ha sido el desarrollo y entrenamiento del agente, ya que ha requerido una investigación previa, tanto para la implementación del agente y entorno como por el uso de aprendizaje por refuerzo.
    \item \textbf{Desarrollo de la web, con la api y el frontend \#6:} La parte de desarrollo de la aplicación web también ha requerido un gran esfuerzo, ya que se ha tenido que desarrollar tanto el backend como el frontend, así como la integración con la API REST y la base de datos.
    \item \textbf{Documentación \#68:}  Ya que se ha tenido que documentar tanto el código como el proyecto en general, así como la memoria y anexos del proyecto.
    \item \textbf{Procesos de CI \#47:} Los procesos de integración continua y calidad de código han sido claves en el inicio del proyecto para asegurar la calidad del código y la correcta ejecución de las pruebas, así como para facilitar el despliegue de la aplicación web.
    \item \textbf{Lecturas de artículos \#10:} La lectura de artículos ha sido necesaria para la investigación previa al desarrollo del proyecto, así como para la comprensión de los conceptos y técnicas utilizados en el proyecto, y la mejora de la función de recompensa del agente.
\end{enumerate}

\section{Estudio de viabilidad}
Realizar un estudio de viabilidad es una parte importante del futuro de un proyecto, ya que permite evaluar si el proyecto es viable desde diferentes perspectivas. En este caso, se ha realizado un estudio de viabilidad económica y legal, en el que se pretende estudiar cúanto dinero hace falta para realizar las tareas del proyecto, y si se cumplen las normativas legales necesarias para su desarrollo y despliegue.

\subsection{Viabilidad económica}

Los costes se dividen principalmente en dos categorías: costes de personal y costes de infraestructura.

Los costes de personal incluyen los salarios de los integrantes del proyecto. En este caso, se ha considerado ún unico empleado, en el cual ha trabajado en el proyecto de forma activa durante más de 5 menes, con una dedicación aproximada de 17 horas semanales, en el que dan unas horas totales de aproxiamdamente 380 horas de trabajo totales.



\subsection{Viabilidad legal}


