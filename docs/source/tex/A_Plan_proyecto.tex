\apendice{Plan de Proyecto Software}

\section{Introducción}



\subsection{Kanban}
La metodología Kanban se ha utilizado para gestionar el flujo de trabajo del proyecto. En el que se usa el tablero de Kanban perdeterminado en los GitHub Projects, donde se han usado las siguientes columnas:
\begin{itemize}
    \item \textbf{Backlog:} En esta columna se encuentran todas las tareas que se han propuesto para el proyecto, pero que aún no se han comenzado a desarrollar, limitado a 5 tareas.
    \item \textbf{Ready:} En esta columna se encuentran todas las tareas que estaban en el "Backlog", y que están listas para ser desarrolladas.
    \item \textbf{In Progress:} En esta columna se encuentran todas las tareas que están en proceso de desarrollo, limitado a 3 tareas.
    \item \textbf{In Review:} En esta columna se encuentran todas las tareas que han sido completadas y están en revisión por parte de los autores del proyecto, limitado a 5 tareas.
    \item \textbf{Done:} En esta columna se encuentran todas las tareas que ya han sido completadas.
\end{itemize}

Todas las tareas que se han terminado quedan registradas en la columna "Done", y se han ido moviendo a lo largo de las columnas según se ha ido avanzando en el desarrollo del proyecto. En cada tarea, si forma parte de desarrollo de una parte de código o documentación, quedan registrado en la tarea una rama de desarrollo de GitHub, y una pull request asociada a la tarea en la que se han incorporado los cambios realizados a la rama principal. En cada pull request se han incluido comentarios creados pro GitHub Copilot, en la que se ha incluido una descripción de los cambios realizados, y una lista de los archivos que han sido modificados, y una revisión de errores o propuestas de mejora para implementar antes de la incorporación de cambios.

\section{Planificación temporal}
Cada semana, si es necesario se actualiza el tablero del proyecto.
Se han planificado cada semana una reunión de seguimiento del proyecto con los tutores, en la que se han revisado las tareas que se han realizado a lo largo de la semana. En la que se plantean propuestas de mejora y correcciones a las tareas realizadas.

Las reuniones semanales empezaron el día 6 de febrero de 2025 hasta el día 5 de junio de 2025, en el que cada jueves a las 10:00 se han realizado las reuniones, a excepción de festivos y vacaciones.

\subsection{Burn down}

\subsection{Diagrama de estado}

\subsecciones

\section{Estudio de viabilidad}

\subsection{Viabilidad económica}

\subsection{Viabilidad legal}


