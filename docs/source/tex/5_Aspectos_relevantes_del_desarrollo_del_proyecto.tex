\capitulo{5}{Aspectos relevantes del desarrollo del proyecto}

Este apartado pretende recoger los aspectos más interesantes del desarrollo del proyecto, comentados por los autores del mismo.
Debe incluir desde la exposición del ciclo de vida utilizado, hasta los detalles de mayor relevancia de las fases de análisis, diseño e implementación.
Se busca que no sea una mera operación de copiar y pegar diagramas y extractos del código fuente, sino que realmente se justifiquen los caminos de solución que se han tomado, especialmente aquellos que no sean triviales.
Puede ser el lugar más adecuado para documentar los aspectos más interesantes del diseño y de la implementación, con un mayor hincapié en aspectos tales como el tipo de arquitectura elegido, los índices de las tablas de la base de datos, normalización y desnormalización, distribución en ficheros3, reglas de negocio dentro de las bases de datos (EDVHV GH GDWRV DFWLYDV), aspectos de desarrollo relacionados con el WWW...
Este apartado, debe convertirse en el resumen de la experiencia práctica del proyecto, y por sí mismo justifica que la memoria se convierta en un documento útil, fuente de referencia para los autores, los tutores y futuros alumnos.


El proyecto se ha dividido en tres partes, cada una de las cuales tiene un objetivo específico:
\begin{itemize}
    \item \textbf{Generación del entorno de simulación y la generación de los datos de entrenamiento:} Cuyo objetivo es la generación artificial de paquetes para simular tráfico atacante y tráfico de benigno en un entorno de dispositivos de internet de las cosas. También tiene como objetivo la creación de un entorno de simulación para el entrenamiento del agente de aprendizaje por refuerzo.
    \item \textbf{Desarrollo del entorno y el agente de aprendizaje por refuerzo:} Incluyendo el ajuste de parámetros, mejora en la función de recompensa, entrenamiento del agente y su posterior evaluación una vez entrenado.
    \item \textbf{Desarrollo de la aplicación web:} Su principal objetivo es la visualización de los resultados obtenidos por el agente, y que los usuarios puedan visualizar de manera sencilla los resultados.
\end{itemize}

\section{Generación del entorno de simulación y la generación de los datos de entrenamiento}

\section{Desarrollo del entorno y el agente de aprendizaje por refuerzo.}

\section{Desarrollo de la aplicación web.}