\apendice{Anexo de sostenibilización curricular}

\section{Introducción}
Este anexo incluirá una reflexión personal del alumnado sobre los aspectos de la sostenibilidad que se abordan en el trabajo.
Se pueden incluir tantas subsecciones como sean necesarias con la intención de explicar las competencias de sostenibilidad adquiridas durante el alumnado y aplicadas al Trabajo de Fin de Grado.

Más información en el documento de la CRUE \url{https://www.crue.org/wp-content/uploads/2020/02/Directrices_Sosteniblidad_Crue2012.pdf}.

Este anexo tendrá una extensión comprendida entre 600 y 800 palabras.




Durante el desarrollo del Trabajo de Fin de Grado, se han podido incluir diferentes aspectos relacionados con el desarrollo sostenible, no limitándose únicamente a los aspectos medioambientales, sino también a los sociales y económicos.

En primer lugar, se ha tenido en cuenta el impacto medioambiental de las tecnologías utilizadas, como el uso de contenedores Docker para la ejecución del proyecto. Esta tecnología permite una mayor eficiencia en el uso de recursos, incluyendo una mayor capa de seguridad y facilitando la escalabilidad de la aplicación, lo que contribuye a una reducción del consumo energético y de utilización de recursos.
Además se han utilizado servicios de terceros en la nube, comprometidos con la sostenibilidad, como es el caso de GitHub~\cite{SosGithub}, Cloudflare~\cite{SosCloudflare} y AWS~\cite{SosAWS}.

\section{Uso sostenible de recursos}

Uno de los aspectos más relevantes del proyecto es el uso eficiente de los recursos y la optimización de los mismos. Para ello, se han implementado diversas estrategias que permiten reducir el consumo de recursos y minimizar el impacto ambiental.

En primer lugar, se ha optado por el uso de Cloudflare como proveedor de servicios de DNS y CDN, lo que permite mejorar las velocidades de carga, dismuniendo el ancho de banda y consultas al servidor de origen, ahorrando tiempo de cómputo. Cloudflare también se compromete con la sostenibilidad y ha implementado diversas iniciativas para reducir su huella de carbono, como el uso de energías renovables en sus centros de datos~\cite{SosCloudflare}. Cloudflare proporciona un informe sobre el impacto de carbono de los servicios usados, aunque no se ha podido obtener el informe para este proyecto en concreto, ya que no se ha alcanzado el umbral mínimo de uso de recursos para que se genere dicho informe. Aunque una de las métricas que si se pueden obtener es el número del ancho de banda ahorrado, que en este caso se muestra en la \textbf{Figura~\ref{fig:ancho}}.
\imagen{ancho}{Ancho de banda ahorrado por el uso Cloudflare}
En el que se puede observar en la \textbf{Figura~\ref{fig:ancho}} que se ha ahorrado un porcentaje considerable, que se traduce en un menor tiempo de cómputo, que deriva en una reducción del consumo energético, del impacto ambiental y reducciones de costes.



