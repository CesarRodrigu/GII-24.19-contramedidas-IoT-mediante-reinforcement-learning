\capitulo{2}{Objetivos del proyecto}

Este apartado explica de forma precisa y concisa cuales son los objetivos que se persiguen con la realización del proyecto. Se puede distinguir entre los objetivos marcados por los requisitos del software a construir y los objetivos de carácter técnico que plantea a la hora de llevar a la práctica el proyecto.

Los principales objetivos del proyecto son:
\begin{itemize}
    \item Implementar un entorno de simulación para el entrenamiento del agente.
    \item Desarrollar un agente de inteligencia artificial, mediante aprendizaje por refuerzo capaz de detectar y mitigar ataques a una red informática, especialmente en un entorno IoT.
    \item Desarrollar y desplegar una aplicación web que permita la visualización de los resultados obtenidos por el agente.
\end{itemize}

\section{Objetivos generales}
El objetivo general del proyecto es el desarrollo de un agente de aprendizaje por refuerzo capaz de mitigar ataques de denegación de servicio basado en la ocupación de la cola de un router de la red, en un entorno de dispositivos de internet de las cosas. Para ello se ha desarollado un entorno de simulación que permite la generación de tráfico benigno y malicioso, así como el entrenamiento del agente. Además, se ha desarrollado una aplicación web que permite la gestión de los modelos entrenados por cada usuario, mostrar la documentación asociada y visualizar los resultados obtenidos por el agente, junto con la visualización de la función de recompensa y el estado del entorno.

\section{Objetivos específicos}
Los objetivos más específicos del proyecto son:

\subsection{Generación del entorno de simulación y la generación de los datos de entrenamiento}
El objetivo de esta parte del proyecto es la generación artificial de generadores de paquetes para simular tráfico benigno y tráfico atacante en un entorno de dispositivos de internet de las cosas, así como la creación de un entorno de simulación para el entrenamiento del agente de aprendizaje por refuerzo. Los requisitos específicos de este apartado son:
\begin{itemize}
    \item Desarrollar generadores de tráfico benigno representativo de dispositivos IoT basado en datos realistas.
    \item Implementar generadores de tráfico malicioso que simulen ataques de denegación de servicio basado en datos realistas.
    \item Realizar un estudio de la función de recompensa para su optimización, en el que se muestre la evolución de la recompensa según la acción y el estado del entorno.
    \item Crear un entorno de simulación que permita el entrenamiento y la evaluación del agente.
    \item Realizar una evaluación gráfica de los resultados obtenidos por el agente.
\end{itemize}

\subsection{Desarrollo de la aplicación web}
El propósito de este apartado es el desarrollo de una aplicación web para la gestión y visualización de los modelos y resultados. Los requisitos específicos son:
\begin{itemize}
    \item Implementar una interfaz para la visualización de algunos los resultados obtenidos por el agente.
    \item Permitir la gestión y almacenamiento persistente de los modelos entrenados por los usuarios.
    \item Incluir la documentación asociada a cada modelo, accesible desde la aplicación.
    \item Facilitar la visualización de la función de recompensa y del estado del entorno de simulación.
    \item Permitir la gestión de los modelos almacenados por cada usuario.
    \item Permitir la gestión de los usuarios por parte del administrador.
    \item Proporcionar ayuda a los usuarios en la aplicación web para facilitar su uso.
\end{itemize}

\subsection{Otros objetivos}
Además de los objetivos ya descritos es importante destacar otros objetivos que se han tenido en cuenta durante el desarrollo del proyecto:
\begin{itemize}
    \item \textbf{Seguridad:} Desarrollar el sistema siguiendo buenas prácticas de seguridad informática para proteger los datos y la integridad del sistema.
    \item \textbf{Reducción de costes:} Diseñar el sistema de forma que se minimicen los recursos requeridos y el costo asociado, especialmente para el despliegue de la aplicación web.
    \item \textbf{Usabilidad:} Diseñar una interfaz web intuitiva, accesible y simple para los usuarios, asegurando una experiencia de usuario satisfactoria.
    \item \textbf{Escalabilidad:} Permitir que el sistema pueda ser fácilemente ampliado o adaptado a nuevas necesidades o requisitos futuros.
    \item \textbf{Mantenibilidad:} Estructurar el código del proyecto de forma modular y documentada para facilitar su mantenimiento, extensión o reutilización en futuros desarrollos, siguiendo patrones de diseño y buenas prácticas de programación.
\end{itemize}