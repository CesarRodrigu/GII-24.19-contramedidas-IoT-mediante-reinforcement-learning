\capitulo{1}{Introducción}

En la actualidad, el uso masivo de dispositivos conectados a Internet ha permitido a los cibercriminales llevar a cabo ataques informáticos de forma más sencilla y efectiva. Estos ataques pueden tener consecuencias graves, como el robo de datos, la interrupción de servicios, el secuestro de sistemas, etc. Por lo que es fundamental tener medidas de seguridad adecuadas para proteger todo sistema informático.

Debido al crecimiento y desarrollo de nuevas tecnologías, ha provocado un aumento en la frecuencia, automatización y complejidad de los ciberataques~\cite{Li2021}. Según el informe de IBM Security, el coste medio global de una brecha de datos alcanzó los 4,9 millones de dólares. Por otro lado, el coste ahorrado en organizaciones que usan seguridad con inteligencia artificial y una automatización extensa en la seguridad es de 2,2 millones de dólares en promedio \footnote{IBM Security. (2024). \textit{Cost of a Data Breach Report 2024}. \url{https://www.ibm.com/reports/data-breach}}. Esto ha generado una necesidad de invertir en la investigación y el desarrollo de nuevas tecnologías para mejorar la seguridad, garantizando el correcto funcionamiento de los sistemas a la par que seguros. También ha generado que todos los profesionales que trabajan con sistemas informáticos estén concienciados de la importancia de la seguridad informática y de la necesidad de seguir e implementar buenas prácticas de seguridad en todos los procesos.

En los últimos años, los dispositivos de Internet de las Cosas (IoT) se han vuelto cada vez más populares, tanto en el ámbito empresarial como en el doméstico. Estos dispositivos, que van desde cámaras de seguridad hasta termostatos inteligentes, ofrecen una gran comodidad y eficiencia en la recolección y gestión de datos. Sin embargo, también son dispositivos muy vulnerables a ciberataques, ya que muchos de ellos carecen de medidas de seguridad adecuadas, por lo que es aún más importante implementar buenas prácticas de seguridad en estos dispositivos.

Otro tema que está en auge en la actualidad es el uso de la Inteligencia Artificial (IA) en múltiples áreas, incluyendo la ciberseguridad. La IA puede ayudar a detectar patrones, predecir amenazas e incluso mitigar ataques en tiempo real.

Se ha extendido el uso del aprendizaje por refuerzo para todo tipo de temas~\cite{Matsuo2022}, incluyendo la robótica, el control autónomo de vehículos y, más recientemente, la ciberseguridad. Este tipo de aprendizaje automático permite a los sistemas realizar acciones basadas en lo que ha aprendido y en una retroalimentación del entorno, convirtiéndolo en un enfoque adecuado para un entorno tan complejo y dinámico como es la ciberseguridad, un entorno que cambia constantemente y con ataques muy diversos.
En lugar de aprender solo de datos históricos, permite adaptarse en tiempo real a nuevas situaciones, optimizando su política de decisión, y ante amenazas desconocidas, permite adaptarse y aprender de ellas. Esta capacidad de adaptación posiciona al aprendizaje por refuerzo como una herramienta clave para la ciberseguridad.

Por lo que en este trabajo se explorará cómo la IA, mediante el aprendizaje por refuerzo, puede mejorar la seguridad de los sistemas informáticos, especialmente en el contexto de los dispositivos IoT.

\section{Estructura del trabajo}
La documentación se ha dividido en 2 partes principales:
\begin{itemize}
    \item \textbf{Memoria.} En esta parte se presenta y describe el trabajo realizado, incluyendo la motivación, objetivos, metodología y resultados obtenidos.
    \item \textbf{Anexos.} Esta parte se centra más en los aspectos técnicos del trabajo del desarrollo del software y elementos importantes en la gestión del proyecto.
\end{itemize}

Todo el trabajo realizado está disponible en el repositorio de GitHub del proyecto que es: \url{https://github.com/CesarRodrigu/GII-24.19-contramedidas-IoT-mediante-reinforcement-learning}. 

También se ha creado un sitio web para el proyecto. El sitio web es: \url{https://cesarrv.com}.

Para el control de calidad del software se ha utilizado SonarQube, una herramienta de análisis de código estático que permite detectar problemas de calidad en el código. El informe de SonarQube se puede consultar en el siguiente enlace: \url{https://sonarcloud.io/project/overview?id=CesarRodrigu_GII-24.19-contramedidas-IoT-mediante-reinforcement-learning}.