\capitulo{1}{Introducción}

En la actualidad, el uso masivo de dispositivos conectados a Internet ha permitido a los cibercriminales llevar a cabo ataques informáticos de forma más sencilla y efectiva. Estos ataques pueden tener consecuencias graves, como el robo de datos, la interrupción de servicios, etc. Por lo que es fundamental tener medidas de seguridad adecuadas para proteger todo sistema informático.

Esto ha generado una necesidad de invertir en la investigación y el desarrollo de nuevas tecnologías para mejorar la seguridad, garantizando el correcto funcionamiento de los sistemas. También ha generado que todos los profesionales que trabajan con sistemas informáticos estén concienciados de la importancia de la seguridad informática y de la necesidad de seguir e implementar buenas prácticas de seguridad en todos los procesos.

En los últimos años, los dispositivos de Internet de las Cosas (IoT) se han vuelto cada vez más populares, tanto en el ámbito empresarial como en el doméstico. Estos dispositivos, que van desde cámaras de seguridad hasta termostatos inteligentes, ofrecen una gran comodidad y eficiencia en la recolección y gestión de datos. Sin embargo, también son dispositivos muy vulnerables a ciberataques, ya que muchos de ellos carecen de medidas de seguridad adecuadas, por lo que es aún más importante implementar buenas prácticas de seguridad en estos dispositivos.

Otro tema que está en auge en la actualidad es el uso de la inteligencia artificial (IA) en múltiples áreas, incluyendo la ciberseguridad. La IA puede ayudar a detectar patrones, predecir amenazas e incluso mitigar ataques en tiempo real. 
Por lo que en este trabajo se explorará cómo la IA, mediante el aprendizaje por refuerzo, puede mejorar la seguridad de los sistemas informáticos, especialmente en el contexto de los dispositivos IoT.

\section{Estructura de la documentación}
La documentación se ha dividido en 2 partes principales:
\begin{itemize}
    \item \textbf{Memoria.} En esta parte se presenta y describe el trabajo realizado, incluyendo la motivación, objetivos, metodología y resultados obtenidos.
    \item \textbf{Anexos.} Esta parte se centra más en los aspectos técnicos del trabajo del desarrollo del software y elementos importantes en la gestión del proyecto.
\end{itemize}

\section{Recursos adicionales}
Todo el trabajo realizado está disponible en el repositorio de GitHub del proyecto que es: \url{https://github.com/CesarRodrigu/GII-24.19-contramedidas-IoT-mediante-reinforcement-learning}. 

También se ha creado un sitio web para el proyecto. El sitio web es: \url{https://cesarrv.com}.

Para el control de calidad del software se ha utilizado SonarQube, una herramienta de análisis de código estático que permite detectar problemas de calidad en el código. El informe de SonarQube se puede consultar en el siguiente enlace: \url{https://sonarcloud.io/project/overview?id=CesarRodrigu_GII-24.19-contramedidas-IoT-mediante-reinforcement-learning}.